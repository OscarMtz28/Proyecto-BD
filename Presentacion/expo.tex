\documentclass{beamer}
\usepackage[spanish]{babel}
\usepackage[utf8]{inputenc}
\usepackage{listings}
\usepackage{xcolor}
\usetheme{Madrid}

% Configuración para código SQL
\lstset{
    language=SQL,
    basicstyle=\ttfamily\small,
    keywordstyle=\color{blue},
    commentstyle=\color{green},
    stringstyle=\color{red},
    showstringspaces=false,
    breaklines=true,
    frame=single,
    numbers=left,
    numberstyle=\tiny\color{gray},
    tabsize=2
}

\title{Conceptos Avanzados en PostgreSQL}
\subtitle{Triggers, Índices y Stored Procedures}
\author{Tu Nombre}
\date{\today}

\begin{document}

\begin{frame}
\titlepage
\end{frame}

\begin{frame}{Contenido}
\tableofcontents
\end{frame}

\section{Introducción}
\begin{frame}{Introducción}
\begin{block}{¿Qué veremos hoy?}
Tres poderosas herramientas de PostgreSQL para mejorar el rendimiento y la funcionalidad de tus bases de datos:
\begin{itemize}
\item Triggers (Disparadores)
\item Índices
\item Stored Procedures (Procedimientos Almacenados)
\end{itemize}
\end{block}
\end{frame}

\section{Triggers}
\begin{frame}{¿Qué son los Triggers?}
\begin{block}{Definición}
Un trigger (disparador) es un procedimiento que se ejecuta automáticamente cuando ocurre un evento específico en una tabla o vista.
\end{block}

\begin{itemize}
\item Se activan antes o después de INSERT, UPDATE, DELETE
\item Pueden ser por fila o por sentencia
\item Útiles para auditoría, integridad compleja, réplicas
\end{itemize}
\end{frame}

\begin{frame}[fragile]{Ejemplo de Trigger}
\begin{lstlisting}
-- Crear una función para el trigger
CREATE OR REPLACE FUNCTION registrar_cambios()
RETURNS TRIGGER AS $$
BEGIN
    IF (TG_OP = 'DELETE') THEN
        INSERT INTO auditoria(tabla, accion, viejo_valor)
        VALUES (TG_TABLE_NAME, 'DELETE', OLD.*);
        RETURN OLD;
    ELSIF (TG_OP = 'UPDATE') THEN
        INSERT INTO auditoria(tabla, accion, viejo_valor, nuevo_valor)
        VALUES (TG_TABLE_NAME, 'UPDATE', OLD.*, NEW.*);
        RETURN NEW;
    ELSIF (TG_OP = 'INSERT') THEN
        INSERT INTO auditoria(tabla, accion, nuevo_valor)
        VALUES (TG_TABLE_NAME, 'INSERT', NEW.*);
        RETURN NEW;
    END IF;
    RETURN NULL;
END;
$$ LANGUAGE plpgsql;
\end{lstlisting}
\end{frame}

\begin{frame}[fragile]{Crear el Trigger}
\begin{lstlisting}
-- Asociar el trigger a una tabla
CREATE TRIGGER trigger_auditoria
AFTER INSERT OR UPDATE OR DELETE ON empleados
FOR EACH ROW EXECUTE FUNCTION registrar_cambios();
\end{lstlisting}

\begin{block}{Ventajas}
\begin{itemize}
\item Automatización de tareas
\item Mantenimiento de la integridad de datos
\item Auditoría de cambios
\end{itemize}
\end{block}
\end{frame}

\section{Índices}
\begin{frame}{¿Qué son los Índices?}
\begin{block}{Definición}
Estructuras de datos que mejoran la velocidad de las operaciones de recuperación en una tabla.
\end{block}

\begin{itemize}
\item Similar al índice de un libro
\item Acelera SELECT pero puede ralentizar INSERT/UPDATE/DELETE
\item PostgreSQL soporta varios tipos: B-tree, Hash, GiST, SP-GiST, GIN, BRIN
\end{itemize}
\end{frame}

\begin{frame}[fragile]{Tipos de Índices}
\begin{lstlisting}
-- Índice B-tree (predeterminado, bueno para comparaciones)
CREATE INDEX idx_empleados_nombre ON empleados(nombre);

-- Índice Hash (solo para igualdad)
CREATE INDEX idx_empleados_email ON empleados USING HASH(email);

-- Índice GIN (bueno para datos compuestos como arrays)
CREATE INDEX idx_productos_etiquetas ON productos USING GIN(etiquetas);

-- Índice parcial (solo para un subconjunto)
CREATE INDEX idx_empleados_activos ON empleados(id) 
WHERE activo = true;
\end{lstlisting}
\end{frame}

\begin{frame}{Cuándo usar Índices}
\begin{block}{Recomendaciones}
\begin{itemize}
\item Columnas usadas frecuentemente en WHERE
\item Columnas usadas en JOINs
\item Columnas con UNIQUE constraints
\item Columnas usadas en ORDER BY
\end{itemize}
\end{block}

\begin{alertblock}{Precaución}
No indexar todo indiscriminadamente. Cada índice ocupa espacio y ralentiza escrituras.
\end{alertblock}
\end{frame}

\section{Stored Procedures}
\begin{frame}{¿Qué son los Stored Procedures?}
\begin{block}{Definición}
Bloques de código SQL almacenados en el servidor que pueden ser invocados por aplicaciones.
\end{block}

\begin{itemize}
\item Escritos en PL/pgSQL (u otros lenguajes como Python, Perl)
\item Pueden contener lógica compleja
\item Mejoran el rendimiento al ejecutarse en el servidor
\end{itemize}
\end{frame}

\begin{frame}[fragile]{Ejemplo de Stored Procedure}
\begin{lstlisting}
CREATE OR REPLACE FUNCTION aumentar_salario(
    porcentaje float,
    depto_id integer
) RETURNS void AS $$
BEGIN
    -- Aumentar salario a empleados del departamento
    UPDATE empleados
    SET salario = salario * (1 + porcentaje/100)
    WHERE departamento_id = depto_id;
    
    -- Registrar el cambio
    INSERT INTO historico_salarios(
        fecha, departamento_id, porcentaje
    ) VALUES (NOW(), depto_id, porcentaje);
END;
$$ LANGUAGE plpgsql;
\end{lstlisting}
\end{frame}

\begin{frame}[fragile]{Llamando un Stored Procedure}
\begin{lstlisting}
-- Llamar al procedimiento
CALL aumentar_salario(5.0, 10);

-- O usando SELECT para funciones que retornan valores
SELECT * FROM obtener_empleados_por_departamento(10);
\end{lstlisting}

\begin{block}{Ventajas}
\begin{itemize}
\item Encapsulación de lógica de negocio
\item Mayor seguridad (ocultan detalles de implementación)
\item Mejor rendimiento (menos viajes cliente-servidor)
\end{itemize}
\end{block}
\end{frame}

\section{Comparación}
\begin{frame}{Comparación de Conceptos}
\begin{table}
\centering
\begin{tabular}{|l|l|l|}
\hline
\textbf{Característica} & \textbf{Triggers} & \textbf{Stored Procedures} \\
\hline
Ejecución & Automática & Manual (llamada explícita) \\
\hline
Propósito & Reacción a eventos & Encapsular lógica \\
\hline
Retorno & Generalmente no & Puede retornar valores \\
\hline
Parámetros & Implícitos (NEW, OLD) & Explícitos \\
\hline
\end{tabular}
\end{table}

\begin{block}{Índices}
Los índices son estructuras pasivas que mejoran el rendimiento, no contienen lógica.
\end{block}
\end{frame}

\section{Conclusión}
\begin{frame}{Conclusión}
\begin{itemize}
\item \textbf{Triggers}: Automatizan acciones ante eventos en la BD
\item \textbf{Índices}: Mejoran el rendimiento de consultas (con trade-offs)
\item \textbf{Stored Procedures}: Encapsulan lógica compleja en el servidor
\end{itemize}

\begin{block}{Recomendación final}
Usa estas herramientas estratégicamente para construir bases de datos más eficientes, seguras y mantenibles.
\end{block}
\end{frame}

\begin{frame}{Preguntas}
\centering
¿Preguntas?
\end{frame}

\end{document}